% !TeX program = tectonic
\documentclass{beamer}

% Theme
\usetheme{Madrid}
\usecolortheme{default}
\setbeamertemplate{navigation symbols}{}

% Packages
% Use fontspec under XeLaTeX/LuaLaTeX for robust Unicode support
\usepackage[utf8]{inputenc}
\usepackage[T1]{fontenc}
\usepackage{lmodern}
% Fontspec removed for Tectonic compatibility
\usepackage{amsmath, amssymb, bm}
\usepackage{graphicx}
\usepackage{booktabs}
\usepackage{array}
\usepackage{arydshln}
\usepackage{hyperref}

% Sanitize PDF bookmarks: replace math/special macros with plain-text fallbacks
\pdfstringdefDisableCommands{%
  \def\textemdash{-}%
  \def\P{P}%
  \def\E{E}%
  \def\Var{Var}%
  \def\ell{ell}%
  \def\mathbb#1{#1}%
  \def\mathcal#1{#1}%
  \def\bm#1{#1}%
}

\usepackage{centernot}
\usepackage{mathtools}
\usepackage{tikz}
\usetikzlibrary{arrows.meta,positioning,calc,decorations.pathmorphing}
\usepackage{adjustbox}


% Define colors
\definecolor{gold}{RGB}{255,215,0}

% Ensure figures auto-fit within slides by default (consistent with Lesson 1)
\setkeys{Gin}{width=\linewidth,height=0.62\textheight,keepaspectratio}

% Notation and helpers (aligned with Lessons 00/01)
\newcommand{\R}{\mathbb{R}}
\renewcommand{\P}{\mathbb{P}}
\newcommand{\E}{\mathbb{E}}
\newcommand{\Var}{\operatorname{Var}}
\newcommand{\Cov}{\operatorname{Cov}}
\newcommand{\MSE}{\operatorname{MSE}}
\newcommand{\Bias}{\operatorname{Bias}}
\newcommand{\1}{\mathbf{1}}
\newcommand{\indep}{\perp\!\!\!\perp}
\newcommand{\toP}{\xrightarrow{\,\mathsf{P}\,}}
\newcommand{\toas}{\xrightarrow{\,\mathsf{a.s.}\,}}
\newcommand{\tod}{\xrightarrow{\,\mathcal{D}\,}}
\newcommand{\I}{I}
% Additional macros for Lesson 3
\newcommand{\MLE}{\text{MLE}}
\newcommand{\MoM}{\text{MoM}}
\newcommand{\CI}{\text{CI}}
\newcommand{\FisherInfo}{\mathcal{I}}
\newcommand{\plim}{\operatorname{plim}}
\newcommand{\LLN}{\text{LLN}}
\newcommand{\ChiSq}{\chi^2}
\newcommand{\Normal}{\mathcal{N}}
\newcommand{\Uniform}{\mathcal{U}}
\newcommand{\Poisson}{\text{Poisson}}
\newcommand{\Exponential}{\text{Exp}}
\newcommand{\Bernoulli}{\text{Bernoulli}}
\newcommand{\median}{\operatorname{median}}

% Helper for robust top-line commands (following safe-latex-edits.md guidelines)
% Define a wrapper that accepts an optional argument so calls like
% `\robustcmd{tableofcontents}[currentsection,...]` work reliably.
\makeatletter
\def\robustcmd#1{%
  \@ifnextchar[{\robustcmd@opt{#1}}{\robustcmd@noopt{#1}}%
}
\def\robustcmd@opt#1[#2]{\csname #1\endcsname[#2]}%
\def\robustcmd@noopt#1{\csname #1\endcsname}%
\makeatother

% Show a mini table of contents at the beginning of each section
\AtBeginSection{%
  % Use \frame{...} here (not the environment) to avoid grouping issues that
  % can sometimes cause empty frames when the optional args or macros are
  % expanded. This explicitly sets the title and then prints the mini-TOC.
  \frame{%
    \frametitle{Outline}
    \centering
    {\Large \insertsection}\par\vspace{0.6em}
    \vfill
    \begin{minipage}{0.92\textwidth}
      % Use the robust alias so macro-based wrappers still work with optional args
      \robustcmd{tableofcontents}[currentsection,hideothersubsections]
    \end{minipage}
  }
}

% Title information (robust wrappers as per safe-latex-edits)
\robustcmd{title}{Lesson 3 --- Estimator Properties}
\robustcmd{subtitle}{Consistency, Bias, Variance, Confidence Intervals}
\robustcmd{author}{Applied Statistics Course}
\robustcmd{date}{}

\begin{document}

\begin{frame}
  \robustcmd{titlepage}
\end{frame}

% --- Section inputs (modularized) ---
% Updated to use new module-based structure matching prompt pack
% Module: Bias--Variance Tradeoff
% This section builds on Lesson 2 (MLE/MoM estimators) and provides foundation for confidence intervals in Lesson 3.
% Uses examples from shared/data/heights_weights_sample.csv where appropriate.

\section{Bias--Variance Tradeoff}

% Title & Objectives
\begin{frame}{Bias--Variance Tradeoff}
  \begin{block}{Learning Objectives}
    \begin{itemize}
      \item Define bias, variance, and mean squared error of estimators
      \item Explain the bias--variance decomposition: $\MSE = \Bias^2 + \Var$
      \item Apply tradeoff concepts to shrinkage estimators and sample variance
      \item Connect to Lesson 2 estimator properties and Lesson 1 sampling distributions
    \end{itemize}
  \end{block}

  \vspace{1em}
  \begin{center}
    \textit{Builds on Lesson 2 (MLE/MoM) foundations}
  \end{center}
\end{frame}

% Point Estimation Refresher (Bias, Variance, MSE)
\begin{frame}{Bias}
  \begin{block}{Definition}
    The bias of an estimator is
    \[\Bias(\hat{\theta}) = \E[\hat{\theta}] - \theta.\]
  \end{block}

  \begin{block}{Intuition}
    Bias measures systematic error: if non-zero the estimator is centered away
    from the true parameter even as we average over repeated samples. In practice
    bias affects accuracy and may require correction or a bias--variance tradeoff.
  \end{block}
\end{frame}

\begin{frame}{Bias --- Visual}
  \begin{center}
    \includegraphics[width=0.85\textwidth]{figures/bias_variance_conceptual.png}
  \end{center}
\end{frame}

\begin{frame}{Variance}
  \begin{block}{Definition}
    The variance of an estimator is
    \[\Var(\hat{\theta}) = \E\big[(\hat{\theta} - \E[\hat{\theta}])^2\big].\]
  \end{block}

  \begin{block}{Intuition}
    Variance measures randomness in the estimator across different samples.
    Low variance means repeated experiments produce similar estimates; high
    variance implies lack of precision. For practitioners, variance controls
    confidence interval width and sample size planning.
  \end{block}
\end{frame}

\begin{frame}{Variance --- Visual}
  \begin{center}
    \includegraphics[width=0.85\textwidth]{figures/bias_variance_conceptual.png}
  \end{center}
\end{frame}

\begin{frame}{MSE and Bias--Variance Tradeoff}
  \begin{block}{Definition}
    The mean squared error of an estimator is
    \[\MSE(\hat{\theta}) = \E\big[(\hat{\theta} - \theta)^2\big].\]
    It decomposes as
    \[\MSE(\hat{\theta}) = \Bias(\hat{\theta})^2 + \Var(\hat{\theta}).\]
  \end{block}

  \begin{block}{Intuition}
    MSE trades off accuracy (bias) against precision (variance). In applied work
    a small bias can be acceptable if it substantially reduces variance and
    thereby improves prediction or interval width.
  \end{block}
\end{frame}

\begin{frame}{MSE and Bias--Variance Tradeoff --- Visual}
  \begin{center}
    \includegraphics[width=1.0\linewidth]{figures/bias_variance_tradeoff.png}
  \end{center}
\end{frame}

% Example A: Shrinkage Mean Estimator
\begin{frame}{Shrinkage Estimator}
  \begin{block}{Definition}
    The shrinkage estimator combines data and prior information:
    \[\delta_\alpha = \alpha \bar{X} + (1-\alpha) \mu_0,\]
    where $\mu_0$ is a prior guess and $\alpha \in [0,1]$ controls shrinkage.
  \end{block}

  \begin{block}{Connection to Lesson 2}
    Builds on MLE/MoM estimators by showing how to incorporate prior information
    when data is limited, using heights data from \texttt{shared/data/heights\_weights\_sample.csv}.
  \end{block}
\end{frame}

\begin{frame}{Shrinkage Estimator --- Visual}
  \begin{center}
    \includegraphics[width=0.95\textwidth]{figures/bias_variance_tradeoff.png}
  \end{center}
  \vspace{-0.3cm}
  \footnotesize Panel 1 shows shrinkage effect; Panel 2 shows MSE decomposition
\end{frame}

% Example B: Sample Variance Estimators
\begin{frame}{Sample Variance Estimators}
  \begin{block}{Definition}
    Two common variance estimators (building on Lesson 2 Normal examples):
    \begin{align*}
    s^2 &= \frac{1}{n-1} \sum_{i=1}^n (X_i - \bar{X})^2 \quad (\text{unbiased}), \\
    \hat{\sigma}^2_{\MLE} &= \frac{1}{n} \sum_{i=1}^n (X_i - \bar{X})^2 \quad (\text{MLE, biased}).
    \end{align*}
  \end{block}

  \begin{block}{Tradeoff Analysis}
    The unbiased estimator has lower bias but higher variance; the MLE has
    higher bias but lower variance, illustrating the bias--variance tradeoff.
  \end{block}
\end{frame}

\begin{frame}{Sample Variance Estimators --- Visual}
  \begin{center}
    \includegraphics[width=0.95\textwidth]{figures/bias_variance_tradeoff.png}
  \end{center}
  \vspace{-0.3cm}
  \footnotesize Panel 3 shows comparison of unbiased vs MLE variance estimators
\end{frame}

% Pitfalls & Heuristics
\begin{frame}{Pitfalls \& Heuristics}
  \begin{block}{Common Pitfalls}
    \begin{itemize}
      \item Overlooking bias in small samples
      \item Assuming unbiased = better (ignores variance)
      \item Not considering the use case (prediction vs estimation)
    \end{itemize}
  \end{block}

  \begin{block}{Practical Heuristics}
    \begin{itemize}
      \item Use unbiased estimators when bias matters most
      \item Consider shrinkage when prior information is reliable
      \item Evaluate estimators on MSE for prediction tasks
      \item Check both bias and variance in simulation studies
    \end{itemize}
  \end{block}
\end{frame}

% Exercises
\begin{frame}{Exercises}
  \begin{enumerate}
    \item Derive the bias of the MLE variance estimator $\hat{\sigma}^2_{\MLE} = \frac{1}{n} \sum (X_i - \bar{X})^2$ for Normal data.
    \item Show that $\MSE(\hat{\theta}) = \Bias(\hat{\theta})^2 + \Var(\hat{\theta})$ using the definition of variance.
    \item Simulate the bias--variance tradeoff for the shrinkage estimator $\delta_\alpha = \alpha \bar{X} + (1-\alpha) \mu_0$ with $\mu_0 = 170$ using heights data.
    \item Compare MSE of $s^2$ vs $\hat{\sigma}^2_{\MLE}$ across different sample sizes $n = 5, 10, 20, 50$.
  \end{enumerate}
\end{frame}

% Summary & References
\begin{frame}{Summary}
  \begin{block}{Key Takeaways}
    \begin{itemize}
      \item Bias measures systematic error; variance measures random error
      \item $\MSE = \Bias^2 + \Var$ shows the fundamental tradeoff
      \item Unbiased estimators aren't always better (consider variance)
      \item Shrinkage can reduce variance at the cost of some bias
      \item Choice depends on context: estimation vs prediction
    \end{itemize}
  \end{block}

  \begin{center}
    \textit{Connects Lesson 2 estimators to practical performance evaluation}
  \end{center}

  \footnotesize
  \begin{itemize}
    \item Wikipedia: Bias--variance tradeoff \url{https://en.wikipedia.org/wiki/Bias\%E2\%80\%93variance_tradeoff}
    \item Casella \& Berger, \textit{Statistical Inference} (Chapter 7)
  \end{itemize}
\end{frame}
% Module: Consistency of Estimators
% This section builds directly on Lesson 1 (LLN, CLT) and Lesson 2 (MLE/MoM estimators).
% References the uniform_max_consistency() function from the appendix.

\section{Consistency}

% Title & Objectives
\begin{frame}{Consistency}
  \begin{block}{Learning Objectives}
    \begin{itemize}
      \item Define consistency (convergence in probability) and strong consistency
      \item Connect consistency to the Law of Large Numbers (Lesson 1)
      \item Identify consistent vs inconsistent estimators
      \item Apply consistency concepts to MLE and MoM estimators (Lesson 2)
    \end{itemize}
  \end{block}

  \vspace{1em}
  \begin{center}
    \textit{Builds on Lesson 1 (LLN/CLT) and Lesson 2 (MLE/MoM)}
  \end{center}
\end{frame}

% Consistency Definition
\begin{frame}{Consistency}
  \begin{block}{Definition}
    An estimator $\hat{\theta}_n$ is consistent for $\theta$ if
    \[\hat{\theta}_n \xrightarrow{\;\mathsf{P}\;} \theta \qquad (n \to \infty),\]
    meaning $\forall \epsilon > 0$, $\P(|\hat{\theta}_n - \theta| > \epsilon) \to 0$.
  \end{block}

  \begin{block}{Intuition}
    With increasing sample size the estimator concentrates around the true
    value. Consistency is a minimal long-run requirement: without it an
    estimator may never learn the truth no matter how much data you collect.
  \end{block}
\end{frame}

\begin{frame}{Consistency --- Visual}
  \begin{center}
    \includegraphics[width=0.95\textwidth]{figures/consistency_demonstration.png}
  \end{center}
  \vspace{-0.3cm}
  \footnotesize Panels show sample mean convergence and visualization of consistency
\end{frame}

% Strong Consistency
\begin{frame}{Strong Consistency}
  \begin{block}{Definition}
    An estimator $\hat{\theta}_n$ is strongly consistent if
    \[\hat{\theta}_n \xrightarrow{\;\mathsf{a.s.}\;} \theta \qquad (n \to \infty),\]
    meaning $\P(\lim_{n\to\infty} \hat{\theta}_n = \theta) = 1$.
  \end{block}

  \begin{block}{Connection to Lesson 1}
    Strong consistency follows from the Strong Law of Large Numbers,
    while weak consistency follows from the Weak Law of Large Numbers.
  \end{block}
\end{frame}

% LLN and Consistency
\begin{frame}{Law of Large Numbers and Consistency}
  \begin{block}{Weak LLN $\to$ Consistency}
    For i.i.d. data with $\E[X_i] = \mu < \infty$:
    \[\bar{X}_n \xrightarrow{\;\mathsf{P}\;} \mu \quad \Rightarrow \quad \bar{X}_n \text{ is consistent for } \mu.\]
  \end{block}

  \begin{block}{Strong LLN $\to$ Strong Consistency}
    Under the same conditions:
    \[\bar{X}_n \xrightarrow{\;\mathsf{a.s.}\;} \mu \quad \Rightarrow \quad \bar{X}_n \text{ is strongly consistent for } \mu.\]
  \end{block}

  \begin{block}{Practical Implication}
    Sample means are consistent estimators of population means, justifying
    the use of $\bar{X}_n$ as an estimator for $\mu$ in large samples.
  \end{block}
\end{frame}

% Example: Sample Mean (Consistent)
\begin{frame}{Example: Sample Mean (Consistent)}
  \begin{block}{Normal Case}
    For $X_i \sim \Normal(\mu, \sigma^2)$ i.i.d.:
    \[\bar{X}_n = \frac{1}{n} \sum_{i=1}^n X_i \xrightarrow{\;\mathsf{P}\;} \mu.\]
  \end{block}

  \begin{block}{Connection to Lesson 1}
    This follows directly from the Weak Law of Large Numbers and the
    Central Limit Theorem, providing the foundation for statistical inference.
  \end{block}

  \begin{block}{Why It Works}
    \begin{itemize}
      \item $\E[\bar{X}_n] = \mu$ (unbiased)
      \item $\Var(\bar{X}_n) = \sigma^2/n \to 0$ (variance vanishes)
      \item By Chebyshev: $\P(|\bar{X}_n - \mu| > \epsilon) \leq \Var(\bar{X}_n)/\epsilon^2 \to 0$
    \end{itemize}
  \end{block}
\end{frame}

% Example: Max for Uniform (Biased but Consistent)
\begin{frame}{Example: Uniform Maximum (Biased but Consistent)}
  \begin{block}{Setup}
    Let $X_i \sim \Uniform[0, \theta]$ i.i.d., and consider
    \[\hat{\theta}_n = \max\{X_1, \dots, X_n\}.\]
  \end{block}

  \begin{block}{Properties}
    \begin{itemize}
      \item $\E[\hat{\theta}_n] = \frac{n}{n+1} \theta < \theta$ (biased)
      \item $\Var(\hat{\theta}_n) \to 0$ as $n \to \infty$
      \item $\hat{\theta}_n \xrightarrow{\;\mathsf{P}\;} \theta$ (consistent)
    \end{itemize}
  \end{block}

  \begin{block}{Connection to Lesson 2}
    This estimator is the MLE for the Uniform$[0,\theta]$ distribution,
    demonstrating that biased estimators can still be consistent.
  \end{block}
\end{frame}

\begin{frame}{Uniform Maximum --- Visual}
  \begin{center}
    \includegraphics[width=0.95\textwidth]{figures/consistency_demonstration.png}
  \end{center}
  \vspace{-0.3cm}
  \footnotesize Panel 4 shows the uniform maximum estimator converging to $\theta$
\end{frame}

% Example: Inconsistent Estimator
\begin{frame}{Example: Inconsistent Estimator}
  \begin{block}{Counterexample}
    Consider using $X_1$ (the first observation) as an estimator for $\mu$:
    \[\hat{\mu}_n = X_1.\]
  \end{block}

  \begin{block}{Why Inconsistent}
    \begin{itemize}
      \item $\E[\hat{\mu}_n] = \mu$ (unbiased)
      \item $\Var(\hat{\mu}_n) = \sigma^2$ (variance doesn't decrease with $n$)
      \item $\P(|\hat{\mu}_n - \mu| > \epsilon) = \P(|X_1 - \mu| > \epsilon) \not\to 0$
    \end{itemize}
  \end{block}

  \begin{block}{Contrast with Lesson 1}
    This violates the variance vanishing requirement for consistency,
    unlike the sample mean which satisfies $\Var(\bar{X}_n) \to 0$.
  \end{block}
\end{frame}

% Pitfalls: Heavy Tails
\begin{frame}{Pitfalls: Heavy-Tailed Distributions}
  \begin{block}{LLN Failure}
    The Law of Large Numbers requires finite variance. For heavy-tailed
    distributions with $\Var(X_i) = \infty$, the sample mean may not be consistent.
  \end{block}

  \begin{block}{Example}
    For $X_i \sim$ Cauchy distribution:
    \begin{itemize}
      \item No finite mean or variance
      \item Sample mean does not converge (in probability or a.s.)
      \item Need robust alternatives (median, trimmed mean)
    \end{itemize}
  \end{block}

  \begin{block}{Practical Implication}
    Always check moment conditions before assuming consistency,
    especially with real data that may have heavy tails.
  \end{block}
\end{frame}

% Exercises
\begin{frame}{Exercises}
  \begin{enumerate}
    \item Prove that if $\hat{\theta}_n \xrightarrow{\;\mathsf{P}\;} \theta$ and $g$ is continuous, then $g(\hat{\theta}_n) \xrightarrow{\;\mathsf{P}\;} g(\theta)$.
    \item Show that the sample median is consistent for the population median under mild conditions.
    \item Use the \texttt{uniform\_max\_consistency()} function from the appendix to verify that $\max\{X_i\}$ is consistent for $\theta$ in Uniform$[0,\theta]$.
    \item Explain why $X_1$ is inconsistent for $\mu$ while $\bar{X}_n$ is consistent.
  \end{enumerate}
\end{frame}

% Summary & References
\begin{frame}{Summary}
  \begin{block}{Key Takeaways}
    \begin{itemize}
      \item Consistency requires both correct centering and vanishing variance
      \item Sample means are consistent by LLN (Lesson 1 foundation)
      \item Biased estimators can be consistent if bias vanishes
      \item MLEs and MoM estimators are typically consistent (Lesson 2)
      \item Heavy tails can break consistency assumptions
    \end{itemize}
  \end{block}

  \begin{center}
    \textit{Provides theoretical foundation for Lesson 2 estimators}
  \end{center}

  \footnotesize
  \begin{itemize}
    \item Wikipedia: Consistent estimator \url{https://en.wikipedia.org/wiki/Consistent_estimator}
    \item Casella \& Berger, \textit{Statistical Inference} (Chapter 7)
  \end{itemize}
\end{frame}
% Module: Asymptotic Efficiency & the Cramér--Rao Lower Bound (CRLB)
% This section extends Lesson 2 (Fisher information, MLE properties) and provides foundation for confidence intervals.
% Uses fisher_info_* functions from the appendix.

\section{Asymptotic Efficiency \& CRLB}

% Title & Objectives
\begin{frame}{Asymptotic Efficiency \& Cramér--Rao Lower Bound}
  \begin{block}{Learning Objectives}
    \begin{itemize}
      \item Define score function, Fisher information, and CRLB for unbiased estimators
      \item State regularity conditions and equality cases
      \item Explain asymptotic normality of MLEs and asymptotic efficiency
      \item Work through Normal, Poisson, and Exponential examples (Lesson 2)
    \end{itemize}
  \end{block}

  \vspace{1em}
  \begin{center}
    \textit{Extends Lesson 2 (Fisher information, MLE) foundations}
  \end{center}
\end{frame}

% Score Function
\begin{frame}{Score Function}
  \begin{block}{Definition}
    The score function is the derivative of the log-likelihood:
    \[U(\theta) = \frac{\partial \ell(\theta)}{\partial \theta}.\]
  \end{block}

  \begin{block}{Intuition}
    The score measures the sensitivity of the log-likelihood to changes
    in the parameter. It indicates the direction and magnitude of the
    gradient that the MLE will follow.
  \end{block}

  \begin{block}{Connection to Lesson 2}
    For i.i.d. data, the total score is $U_n(\theta) = \sum_{i=1}^n U_i(\theta)$,
    and the MLE satisfies $U_n(\hat{\theta}_{\MLE}) = 0$.
  \end{block}
\end{frame}

\begin{frame}{Score Function --- Visual}
  \begin{center}
    \includegraphics[width=0.95\textwidth]{figures/fisher_information_visualization.png}
  \end{center}
  \vspace{-0.3cm}
  \footnotesize Panels show likelihood curves and Fisher information for different distributions
\end{frame}

% Fisher Information
\begin{frame}{Fisher Information}
  \begin{block}{Definition}
    The Fisher information is the variance of the score:
    \[\FisherInfo(\theta) = \Var(U(\theta)) = -\E\left[\frac{\partial^2 \ell(\theta)}{\partial \theta^2}\right].\]
  \end{block}

  \begin{block}{Intuition}
    Fisher information quantifies the amount of information about $\theta$
    contained in the data. Higher information means sharper likelihood
    peaks and lower achievable variance.
  \end{block}

  \begin{block}{Connection to Lesson 2}
    For i.i.d. data, the total information is $\FisherInfo_n(\theta) = n \FisherInfo(\theta)$,
    explaining the $1/n$ scaling of MLE variance.
  \end{block}
\end{frame}

% CRLB
\begin{frame}{Cramér--Rao Lower Bound}
  \begin{block}{Statement}
    For an unbiased estimator $\hat{\theta}$ of $\theta$:
    \[\Var(\hat{\theta}) \geq \frac{1}{\FisherInfo_n(\theta)},\]
    with equality if and only if $\hat{\theta}$ is the MLE (under regularity conditions).
  \end{block}

  \begin{block}{Intuition}
    The CRLB sets the theoretical minimum variance any unbiased estimator
    can achieve. It represents the best possible precision given the data
    and model structure.
  \end{block}

  \begin{block}{Practical Value}
    The CRLB helps set expectations about estimator performance and
    identifies when an estimator is statistically efficient.
  \end{block}
\end{frame}

\begin{frame}{Cramér--Rao Lower Bound --- Visual}
  \begin{center}
    \includegraphics[width=0.95\textwidth]{figures/crlb_achievement.png}
  \end{center}
  \vspace{-0.3cm}
  \footnotesize Shows CRLB achievement for Normal, Poisson, and Exponential distributions
\end{frame}

% Example A: Normal Mean
\begin{frame}{Example: Normal Mean ($\sigma$ known)}
  \begin{block}{Setup}
    $X_i \sim \Normal(\mu, \sigma^2)$ i.i.d. with $\sigma$ known.
    Estimator: $\hat{\mu} = \bar{X}_n$.
  \end{block}

  \begin{block}{Fisher Information}
    \[\FisherInfo(\mu) = \frac{1}{\sigma^2}, \quad \FisherInfo_n(\mu) = \frac{n}{\sigma^2}.\]
  \end{block}

  \begin{block}{CRLB and Efficiency}
    \[\Var(\bar{X}_n) = \frac{\sigma^2}{n} = \frac{1}{\FisherInfo_n(\mu)}.\]
    The sample mean achieves the CRLB and is efficient.
  \end{block}

  \begin{block}{Connection to Lesson 1}
    This follows from the CLT: $\sqrt{n}(\bar{X}_n - \mu) \to \Normal(0, \sigma^2)$.
  \end{block}
\end{frame}

\begin{frame}{Normal Mean --- Visual}
  \begin{center}
    \includegraphics[width=0.95\textwidth]{figures/crlb_achievement.png}
  \end{center}
  \vspace{-0.3cm}
  \footnotesize Panel 1 shows Normal mean estimator achieving CRLB
\end{frame}

% Example B: Poisson Rate
\begin{frame}{Example: Poisson Rate}
  \begin{block}{Setup}
    $X_i \sim \Poisson(\lambda)$ i.i.d.
    Estimator: $\hat{\lambda} = \bar{X}_n$ (both MLE and MoM).
  \end{block}

  \begin{block}{Fisher Information}
    \[\FisherInfo(\lambda) = \frac{1}{\lambda}, \quad \FisherInfo_n(\lambda) = \frac{n}{\lambda}.\]
  \end{block}

  \begin{block}{CRLB and Efficiency}
    \[\Var(\bar{X}_n) = \frac{\lambda}{n} = \frac{1}{\FisherInfo_n(\lambda)}.\]
    The sample mean achieves the CRLB and is efficient.
  \end{block}

  \begin{block}{Connection to Lesson 2}
    This is the same estimator derived by both MLE and MoM methods,
    demonstrating efficiency of both approaches for this model.
  \end{block}
\end{frame}

% Example C: Exponential Rate
\begin{frame}{Example: Exponential Rate}
  \begin{block}{Setup}
    $X_i \sim \Exponential(\lambda)$ i.i.d. (rate parameter).
    MLE: $\hat{\lambda}_{\MLE} = n / \sum_{i=1}^n X_i$.
  \end{block}

  \begin{block}{Fisher Information}
    \[\FisherInfo(\lambda) = \frac{1}{\lambda^2}, \quad \FisherInfo_n(\lambda) = \frac{n}{\lambda^2}.\]
  \end{block}

  \begin{block}{CRLB}
    \[\Var(\hat{\lambda}) \geq \frac{\lambda^2}{n}.\]
  \end{block}

  \begin{block}{Asymptotic Efficiency}
    The MLE achieves the CRLB asymptotically, but may have finite-sample bias.
  \end{block}
\end{frame}

\begin{frame}{Exponential Rate --- Visual}
  \begin{center}
    \includegraphics[width=0.95\textwidth]{figures/crlb_achievement.png}

    \vspace{-0.3cm}
    \footnotesize Panel 3 shows Exponential rate MLE approaching CRLB; Panel 4 compares efficiency
  \end{center}
\end{frame}

% Asymptotic Normality of MLE
\begin{frame}{Asymptotic Normality of MLE}
  \begin{block}{Statement}
    Under regularity conditions:
    \[\sqrt{n}(\hat{\theta}_{\MLE} - \theta_0) \xrightarrow{\;\mathcal{D}\;} \Normal\left(0, \FisherInfo(\theta_0)^{-1}\right).\]
  \end{block}

  \begin{block}{Intuition}
    For large samples, MLEs behave like normal random variables with
    variance equal to the inverse Fisher information.
  \end{block}

  \begin{block}{Practical Implication}
    This justifies the use of normal approximations for MLE confidence
    intervals and hypothesis tests in large samples.
  \end{block}
\end{frame}

% Asymptotic Efficiency
\begin{frame}{Asymptotic Efficiency}
  \begin{block}{Definition}
    An estimator is asymptotically efficient if it achieves the CRLB
    as $n \to \infty$, i.e.,
    \[\sqrt{n}(\hat{\theta}_n - \theta) \xrightarrow{\;\mathcal{D}\;} \Normal\left(0, \FisherInfo(\theta)^{-1}\right).\]
  \end{block}

  \begin{block}{MLE Property}
    Maximum likelihood estimators are asymptotically efficient under
    regularity conditions, making them the gold standard for large samples.
  \end{block}

  \begin{block}{Connection to Lesson 2}
    This explains why MLEs are preferred when sample sizes are large
    and regularity conditions hold.
  \end{block}
\end{frame}

% Pitfalls
\begin{frame}{Pitfalls \& Regularity Conditions}
  \begin{block}{Regularity Conditions}
    \begin{itemize}
      \item Support of distribution doesn't depend on $\theta$
      \item Likelihood differentiable in $\theta$
      \item Fisher information finite and positive
      \item Dominated convergence for expectation interchanges
    \end{itemize}
  \end{block}

  \begin{block}{When CRLB Fails}
    \begin{itemize}
      \item Boundary parameters (e.g., variance near 0)
      \item Discrete parameters with small support
      \item Model misspecification
    \end{itemize}
  \end{block}
\end{frame}

% Exercises
\begin{frame}{Exercises}
  \begin{enumerate}
    \item Compute the Fisher information for Bernoulli$(p)$ and derive the CRLB for unbiased estimators of $p$.
    \item Show that the sample mean achieves the CRLB for Normal$(\mu, \sigma^2)$ with $\sigma$ known.
    \item Use the \texttt{fisher\_info\_poisson()} function from the appendix to compute information for $\lambda = 2, 5, 10$.
    \item Explain why the MLE for Uniform$[0, \theta]$ achieves the CRLB while other estimators may not.
  \end{enumerate}
\end{frame}

% Summary & References
\begin{frame}{Summary}
  \begin{block}{Key Takeaways}
    \begin{itemize}
      \item Fisher information quantifies precision: $I(\theta) = \Var(U(\theta))$
      \item CRLB sets minimum variance: $\Var(\hat{\theta}) \geq 1/I_n(\theta)$
      \item MLEs achieve CRLB asymptotically (asymptotically efficient)
      \item Normal, Poisson, and Exponential examples illustrate efficiency
      \item Regularity conditions ensure CRLB applicability
    \end{itemize}
  \end{block}

  \begin{center}
    \textit{Provides efficiency foundation for Lesson 2 estimators}
  \end{center}

  \footnotesize
  \begin{itemize}
    \item Wikipedia: Fisher information \url{https://en.wikipedia.org/wiki/Fisher_information}
    \item Casella \& Berger, \textit{Statistical Inference} (Chapter 7)
  \end{itemize}
\end{frame}
% Module: Confidence Intervals (CIs)
% This section applies Lesson 1 (CLT) and Lesson 2 (delta method) to construct intervals.
% Uses shared/data/ab_test_clicks.csv for examples and ci_prop_* functions from appendix.

\section{Confidence Intervals}

% Title & Objectives
\begin{frame}{Confidence Intervals}
  \begin{block}{Learning Objectives}
    \begin{itemize}
      \item Define $(1-\alpha)$ confidence intervals and interpret correctly
      \item Derive classical CIs: Normal mean, t-intervals, variance, proportions
      \item Compare proportion CI methods using A/B testing data
      \item Understand pivots and asymptotic CIs (delta method)
    \end{itemize}
  \end{block}

  \vspace{1em}
  \begin{center}
    \textit{Applies Lesson 1 (CLT) and Lesson 2 (delta method)}
  \end{center}
\end{frame}

% Definition and Interpretation
\begin{frame}{Confidence Interval Definition}
  \begin{block}{Definition}
    A $(1-\alpha)$ confidence interval for $\theta$ is an interval $[L_n, U_n]$ such that
    \[\P_{\theta}(L_n \leq \theta \leq U_n) = 1-\alpha \quad \forall \theta.\]
  \end{block}

  \begin{block}{Interpretation}
    If we construct the interval many times, it will contain the true
    parameter in $(1-\alpha) \times 100\%$ of cases. This is a statement
    about the procedure, not any particular interval.
  \end{block}

  \begin{block}{Common Misconception}
    A 95\% CI does NOT mean there's a 95\% probability that the true
    parameter is in \textit{this particular} interval. The true parameter
    is fixed; the interval is random. We're 95\% confident in the
    \textit{procedure}, not in any single interval.
  \end{block}
\end{frame}

\begin{frame}{Confidence Interval --- Visual}
  \begin{center}
    \includegraphics[width=0.7\textwidth]{figures/ci_interpretation.png}
  \end{center}
\end{frame}

% Pivotal Method
\begin{frame}{Pivotal Method}
  \begin{block}{General Approach}
    A pivot is a function $P_n(\theta, X_1,\dots,X_n)$ whose distribution
    doesn't depend on $\theta$. Common pivots:
    \begin{itemize}
      \item $Z = \sqrt{n}(\bar{X}_n - \mu)/\sigma \sim \Normal(0,1)$
      \item $T = \sqrt{n}(\bar{X}_n - \mu)/S \sim t_{n-1}$
      \item $\chi^2 = (n-1)S^2 / \sigma^2 \sim \ChiSq_{n-1}$
    \end{itemize}
  \end{block}

  \begin{block}{CI Construction}
    Find $q_{\alpha/2}, q_{1-\alpha/2}$ such that $\P(q_{\alpha/2} \leq P_n \leq q_{1-\alpha/2}) = 1-\alpha$.
    Then solve for $\theta$ in the inequality.
  \end{block}
\end{frame}

% Normal Mean (sigma known)
\begin{frame}{Normal Mean CI ($\sigma$ known)}
  \begin{block}{Z-Interval}
    For $X_i \sim \Normal(\mu, \sigma^2)$ with $\sigma$ known:
    \[\bar{X}_n \pm z_{1-\alpha/2} \frac{\sigma}{\sqrt{n}}.\]
  \end{block}

  \begin{block}{Derivation}
    \[\P\left(\bar{X}_n - z_{1-\alpha/2} \frac{\sigma}{\sqrt{n}} \leq \mu \leq \bar{X}_n + z_{1-\alpha/2} \frac{\sigma}{\sqrt{n}}\right) = 1-\alpha.\]
  \end{block}

  \begin{block}{Connection to Lesson 1}
    This follows directly from the Central Limit Theorem and justifies
    normal-based inference for large samples.
  \end{block}
\end{frame}

% Normal Mean (sigma unknown) - t-interval
\begin{frame}{Normal Mean CI ($\sigma$ unknown)}
  \begin{block}{t-Interval}
    For $X_i \sim \Normal(\mu, \sigma^2)$ with $\sigma$ unknown:
    \[\bar{X}_n \pm t_{n-1, 1-\alpha/2} \frac{S}{\sqrt{n}},\]
    where $S^2 = \frac{1}{n-1} \sum_{i=1}^n (X_i - \bar{X}_n)^2$.
  \end{block}

  \begin{block}{Pivot}
    The t-statistic $T = \sqrt{n}(\bar{X}_n - \mu)/S$ follows $t_{n-1}$ distribution,
    accounting for uncertainty in the standard error estimate.
  \end{block}

  \begin{block}{Small Sample Performance}
    t-intervals maintain nominal coverage even for small samples when
    normality holds, unlike normal approximations.
  \end{block}
\end{frame}

% Normal Variance CI
\begin{frame}{Normal Variance CI}
  \begin{block}{Chi-Squared Interval}
    For $X_i \sim \Normal(\mu, \sigma^2)$ with $\mu$ unknown:
    \[\left( \frac{(n-1)S^2}{\chi^2_{n-1, 1-\alpha/2}}, \frac{(n-1)S^2}{\chi^2_{n-1, \alpha/2}} \right).\]
  \end{block}

  \begin{block}{Pivot}
    The chi-squared statistic $\chi^2 = (n-1)S^2 / \sigma^2$ follows $\ChiSq_{n-1}$
    distribution, providing the basis for variance intervals.
  \end{block}

  \begin{block}{Note}
    This interval is not symmetric around $S^2$ due to the skewness
    of the chi-squared distribution.
  \end{block}
\end{frame}

% Binomial Proportion CIs
\begin{frame}{Binomial Proportion CIs}
  \begin{block}{Wald Interval}
    \[\hat{p} \pm z_{1-\alpha/2} \sqrt{\frac{\hat{p}(1-\hat{p})}{n}}.\]
    Simple but poor coverage for small $n$ or extreme $p$.
  \end{block}

  \begin{block}{Wilson Score Interval}
    \[ \frac{\hat{p} + z^2/(2n)}{1 + z^2/n} \pm \frac{z}{1 + z^2/n} \sqrt{\frac{\hat{p}(1-\hat{p})}{n} + z^2/(4n^2)}. \]
    Better coverage but more complex formula.
  \end{block}

  \begin{block}{Connection to A/B Testing}
    Use \texttt{shared/data/ab\_test\_clicks.csv} to compare methods with real data.
  \end{block}
\end{frame}

\begin{frame}{Proportion CI Comparison --- Visual}
  \begin{center}
    \includegraphics[width=0.9\textwidth]{figures/proportion_ci_coverage.png}
  \end{center}
\end{frame}

% Asymptotic CIs via Delta Method
\begin{frame}{Asymptotic CIs via Delta Method}
  \begin{block}{Delta Method Statement}
    If $\sqrt{n}(\hat{\theta}_n - \theta) \to \Normal(0, \sigma^2)$, then
    for measurable function $g$,
    \[\sqrt{n}(g(\hat{\theta}_n) - g(\theta)) \to \Normal(0, [g'(\theta)]^2 \sigma^2).\]
  \end{block}

  \begin{block}{CI Construction}
    Approximate CI for $g(\theta)$:
    \[g(\hat{\theta}_n) \pm z_{1-\alpha/2} \sqrt{[g'(\hat{\theta}_n)]^2 \widehat{\Var}(\hat{\theta}_n)/n}.\]
  \end{block}

  \begin{block}{Connection to Lesson 2}
    The delta method extends asymptotic normality of MLEs to functions
    of parameters, enabling inference for ratios, logs, etc.
  \end{block}
\end{frame}

% Delta Method Example: Log-Odds
\begin{frame}{Delta Method Example: Log-Odds}
  \begin{block}{Problem}
    Estimate CI for log-odds $\theta = \log\left(\frac{p}{1-p}\right)$ from $\hat{p} = X/n$ where $X \sim \text{Binomial}(n, p)$.
  \end{block}

  \begin{block}{Solution Steps}
    \begin{enumerate}
      \item For proportion $\hat{p}$: $\sqrt{n}(\hat{p} - p) \to \Normal(0, p(1-p))$
      \item Transformation: $g(p) = \log\left(\frac{p}{1-p}\right)$
      \item Derivative: $g'(p) = \frac{1}{p(1-p)}$
      \item Delta method: $\sqrt{n}(g(\hat{p}) - g(p)) \to \Normal\left(0, \frac{1}{p(1-p)}\right)$
    \end{enumerate}
  \end{block}

  \begin{block}{95\% CI for Log-Odds}
    \[\log\left(\frac{\hat{p}}{1-\hat{p}}\right) \pm 1.96 \sqrt{\frac{1}{n\hat{p}(1-\hat{p})}}.\]
  \end{block}
\end{frame}

\begin{frame}{Delta Method --- Visual}
  \begin{center}
    \includegraphics[width=0.95\textwidth]{figures/delta_method_illustration.png}
  \end{center}
\end{frame}

% Practical Advice
\begin{frame}{Practical Advice}
  \begin{block}{When to Use Each Method}
    \begin{itemize}
      \item Z-intervals: Large samples, known variance
      \item t-intervals: Small samples, unknown variance, normality
      \item Wilson intervals: Proportions, especially small n or extreme p
      \item Delta method: Functions of parameters, large samples
    \end{itemize}
  \end{block}

  \begin{block}{General Guidelines}
    \begin{itemize}
      \item Check assumptions (normality, sample size)
      \item Compare interval width vs coverage
      \item Use simulation to verify performance
      \item Consider bootstrap for complex scenarios
    \end{itemize}
  \end{block}
\end{frame}

% Exercises
\begin{frame}{Exercises}
  \begin{enumerate}
    \item Derive the t-interval for Normal mean with unknown variance using the pivotal method.
    \item Show that the Wilson interval for p=0.5 and large n reduces to the Wald interval.
    \item Use A/B testing data from \texttt{shared/data/ab\_test\_clicks.csv} to compute and compare proportion CIs.
    \item Apply the delta method to construct a CI for the coefficient of variation $\sigma/\mu$.
  \end{enumerate}
\end{frame}

% Summary & References
\begin{frame}{Summary}
  \begin{block}{Key Takeaways}
    \begin{itemize}
      \item CIs quantify uncertainty in parameter estimates
      \item Pivotal method provides general CI construction framework
      \item t-intervals improve on normal approximations for small samples
      \item Wilson intervals provide better proportion coverage than Wald
      \item Delta method enables inference for parameter functions
    \end{itemize}
  \end{block}

  \begin{center}
    \textit{Foundation for hypothesis testing and decision making}
  \end{center}

  \footnotesize
  \begin{itemize}
    \item Wikipedia: Confidence interval \url{https://en.wikipedia.org/wiki/Confidence_interval}
    \item Casella \& Berger, \textit{Statistical Inference} (Chapter 9)
  \end{itemize}
\end{frame}
% Module: Bootstrap (Nonparametric & Parametric)
% This section complements confidence intervals and builds on Lesson 1 (sampling distributions) and Lesson 2 (parametric estimation).
% Uses bootstrap_stat() function from appendix and shared/data/heights_weights_sample.csv for examples.

\section{Bootstrap}

% Title & Objectives
\begin{frame}{Bootstrap}
  \begin{block}{Learning Objectives}
    \begin{itemize}
      \item Explain the bootstrap idea: resampling to approximate sampling distributions
      \item Implement percentile and basic bootstrap confidence intervals
      \item Apply bootstrap to non-smooth statistics (median, quantiles)
      \item Compare bootstrap CI coverage to parametric methods
    \end{itemize}
  \end{block}

  \vspace{1em}
  \begin{center}
    \textit{Complements confidence intervals, builds on Lesson 1 \& 2}
  \end{center}
\end{frame}

% Motivation
\begin{frame}{Motivation}
  \begin{block}{When Analytic Methods Fail}
    \begin{itemize}
      \item Complex statistics (medians, quantiles, correlations)
      \item Non-standard distributions or small samples
      \item Model misspecification or unknown sampling distributions
      \item Functions of multiple parameters
    \end{itemize}
  \end{block}

  \begin{block}{Bootstrap Idea}
    Use the data itself as an estimate of the population distribution.
    Resample with replacement to approximate the sampling distribution
    of any statistic of interest.
  \end{block}

  \begin{block}{Connection to Lesson 1}
    Bootstrap approximates the sampling distribution without knowing
    the true population parameters or distribution form.
  \end{block}
\end{frame}

% Nonparametric Bootstrap Algorithm
\begin{frame}{Nonparametric Bootstrap Algorithm}
  \begin{block}{Steps}
    \begin{enumerate}
      \item Compute statistic of interest: $\hat{\theta} = g(X_1, \dots, X_n)$
      \item For $b = 1$ to $B$:
      \begin{itemize}
        \item Draw bootstrap sample $X_1^*, \dots, X_n^* \sim$ empirical distribution
        \item Compute $\hat{\theta}^{*b} = g(X_1^{*b}, \dots, X_n^{*b})$
      \end{itemize}
      \item Use $\{\hat{\theta}^{*1}, \dots, \hat{\theta}^{*B}\}$ for inference
    \end{enumerate}
  \end{block}

  \begin{block}{Key Insight}
    The empirical distribution $\hat{F}_n$ serves as a nonparametric
    estimate of the true distribution $F$, enabling resampling without
    parametric assumptions.
  \end{block}
\end{frame}

\begin{frame}{Bootstrap Algorithm -- Visual}
  \begin{center}
    \includegraphics[width=0.95\textwidth]{figures/bootstrap_algorithm_visual.png}
  \end{center}
\end{frame}

% Bootstrap Confidence Intervals
\begin{frame}{Bootstrap Confidence Intervals}
  \begin{block}{Percentile Interval}
    \[\left[ q_{\alpha/2}^*, q_{1-\alpha/2}^* \right],\]
    where $q_p^*$ is the $p$-quantile of bootstrap replicates.
  \end{block}

  \begin{block}{Basic Interval}
    \[\left[ 2\hat{\theta} - q_{1-\alpha/2}^*, \, 2\hat{\theta} - q_{\alpha/2}^* \right].\]
    Centers the interval around $\hat{\theta}$ and uses bootstrap variance.
  \end{block}

  \begin{block}{BCa Interval (Advanced)}
    Bias-corrected and accelerated interval that adjusts for bias and
    skewness in the bootstrap distribution.
  \end{block}
\end{frame}

% Example A: Median CI
\begin{frame}{Example: Median CI (Exponential Data)}
  \begin{block}{Setup}
    $X_i \sim \Exponential(\lambda)$ i.i.d. (skewed distribution).
    Statistic: sample median $\hat{\theta} = \median(X_1, \dots, X_n)$.
  \end{block}

  \begin{block}{Challenge}
    The sampling distribution of the median is complex and depends
    on the underlying distribution, making analytic CIs difficult.
  \end{block}

  \begin{block}{Bootstrap Solution}
    Use nonparametric bootstrap to approximate the sampling distribution
    of the median and construct percentile or basic intervals.
  \end{block}
\end{frame}

\begin{frame}{Median Bootstrap --- Visual}
  \begin{center}
    \includegraphics[width=0.9\textwidth]{figures/bootstrap_median_distribution.png}
  \end{center}
\end{frame}

% Example B: Difference in Means
\begin{frame}{Example: Difference in Means}
  \begin{block}{A/B Testing Scenario}
    Compare means between two groups with potentially different variances.
    Use heights data from \texttt{shared/data/heights\_weights\_sample.csv}.
  \end{block}

  \begin{block}{Challenge}
    Welch's t-test assumes normality and provides analytic intervals,
    but bootstrap offers a robust alternative without strong assumptions.
  \end{block}

  \begin{block}{Bootstrap Approach}
    Bootstrap both groups separately and compute bootstrap distribution
    of the difference in means for inference.
  \end{block}
\end{frame}

% Number of Resamples and Practical Considerations
\begin{frame}{Practical Considerations}
  \begin{block}{Number of Bootstrap Resamples}
    \begin{itemize}
      \item $B = 1000$ often sufficient for basic intervals
      \item $B = 5000$ or more for precise quantile estimation
      \item Computational cost scales with $B \times n$
      \item Parallelization can speed up computation
    \end{itemize}
  \end{block}

  \begin{block}{Random Seeds}
    Always set random seeds for reproducibility:
    \begin{itemize}
      \item Different seeds can give slightly different results
      \item Report seeds in publications and assignments
      \item Use \texttt{rng = np.random.default\_rng(2025)}
    \end{itemize}
  \end{block}
\end{frame}

% BCa Method (Conceptual)
\begin{frame}{BCa Method (Conceptual)}
  \begin{block}{Bias Correction}
    Adjusts for bias in the bootstrap distribution when the statistic
    is not centered at the true parameter.
  \end{block}

  \begin{block}{Acceleration}
    Adjusts for skewness and heteroscedasticity in the bootstrap
    distribution using jackknife estimates.
  \end{block}

  \begin{block}{When Helpful}
    BCa intervals often provide better coverage than percentile intervals,
    especially for skewed distributions or small samples.
  \end{block}
\end{frame}

% BCa Method -- Visual Example
\begin{frame}{BCa Method -- Visual Example}
  \begin{center}
    \includegraphics[width=0.98\textwidth]{figures/bca_method_visual.png}
  \end{center}
\end{frame}

% Pitfalls and Limitations
\begin{frame}{Pitfalls and Limitations}
  \begin{block}{When Bootstrap Fails}
    \begin{itemize}
      \item Dependent data (requires block bootstrap or other methods)
      \item Very small samples ($n < 10$) may not work well
      \item Heavy-tailed distributions may need many resamples
      \item Boundary parameters (variances, correlations) need care
    \end{itemize}
  \end{block}

  \begin{block}{Best Practices}
    \begin{itemize}
      \item Always compare to parametric methods when available
      \item Check bootstrap distribution shape for anomalies
      \item Use multiple random seeds to assess stability
      \item Consider parametric bootstrap when model is trusted
    \end{itemize}
  \end{block}
\end{frame}

% Exercises
\begin{frame}{Exercises}
  \begin{enumerate}
    \item Use bootstrap to construct a 95\% confidence interval for the median of exponential data.
    \item Compare bootstrap CI for mean difference vs Welch's t-interval using heights data from \texttt{shared/data/heights\_weights\_sample.csv}.
    \item Implement studentized bootstrap for the sample mean and compare to percentile bootstrap.
    \item Investigate how bootstrap performance degrades with very small sample sizes ($n = 5, 10$).
  \end{enumerate}
\end{frame}

% Summary & References
\begin{frame}{Summary}
  \begin{block}{Key Takeaways}
    \begin{itemize}
      \item Bootstrap approximates sampling distributions via resampling
      \item Percentile and basic intervals are most common
      \item Bootstrap excels for complex statistics (medians, quantiles)
      \item Provides robust alternative to parametric methods
      \item Requires careful consideration of sample size and dependence
    \end{itemize}
  \end{block}

  \begin{center}
    \textit{Robust complement to parametric confidence intervals}
  \end{center}

  \footnotesize
  \begin{itemize}
    \item Wikipedia: Bootstrap (statistics) \url{https://en.wikipedia.org/wiki/Bootstrapping_(statistics)}
    \item Efron \& Tibshirani, \textit{An Introduction to the Bootstrap}
  \end{itemize}
\end{frame}

\end{document}