% Part II — Properties of MLE Estimators
\section{Properties of MLE}

% Slide 8 — MLE Properties
\begin{frame}{Properties of MLE}
  \begin{block}{Summary (formal)}
    The maximum likelihood estimator satisfies, under regularity conditions:
    \begin{itemize}
      \item Consistency
      \item Asymptotic normality
      \item Asymptotic efficiency (attains CRLB)
      \item Invariance: $g(\hat{\theta}_{\text{MLE}})$ is the MLE of $g(\theta)$
    \end{itemize}
  \end{block}

  \begin{block}{Intuition / Relevance}
    MLEs are often the default in applied work because they combine good
    theoretical properties with practical performance for moderate-to-large
    sample sizes; the invariance property makes transforming estimates easy.
  \end{block}

  \vspace{0.7em}
  % Visual moved to a dedicated frame
\end{frame}

% Visual frame — MLE Properties (visual)
\begin{frame}{Properties of MLE -- Visual}
  \begin{center}
    \begin{adjustbox}{max height=0.40\textheight}
    \begin{tikzpicture}[scale=0.9]
      % Crown for MLE
      \node[draw, circle, fill=gold!30, minimum size=2cm] at (0,0) {\textbf{MLE}};
      \draw[thick, gold] (-0.5, 0.7) -- (-0.3, 1.2) -- (0, 0.9) -- (0.3, 1.2) -- (0.5, 0.7);
      \draw[thick, gold] (-0.2, 1.0) -- (0, 1.4) -- (0.2, 1.0);

      % Properties around it
      \node[fill=green!20, rounded corners] at (-2.5, 1) {Consistent};
      \node[fill=blue!20, rounded corners] at (2.5, 1) {Asymp. Normal};
      \node[fill=red!20, rounded corners] at (-2.5, -1) {Efficient};
      \node[fill=purple!20, rounded corners] at (2.5, -1) {Invariant};

      \node at (0, -2) {\textbf{Gold Standard for Large Samples}};
    \end{tikzpicture}
    \end{adjustbox}
  \end{center}
\end{frame}