% Module: Bootstrap (Nonparametric & Parametric)
% This section complements confidence intervals and builds on Lesson 1 (sampling distributions) and Lesson 2 (parametric estimation).
% Uses bootstrap_stat() function from appendix and shared/data/heights_weights_sample.csv for examples.

\section{Bootstrap}

% Title & Objectives
\begin{frame}{Bootstrap}
  \begin{block}{Learning Objectives}
    \begin{itemize}
      \item Explain the bootstrap idea: resampling to approximate sampling distributions
      \item Implement percentile and basic bootstrap confidence intervals
      \item Apply bootstrap to non-smooth statistics (median, quantiles)
      \item Compare bootstrap CI coverage to parametric methods
    \end{itemize}
  \end{block}

  \vspace{1em}
  \begin{center}
    \textit{Complements confidence intervals, builds on Lesson 1 \& 2}
  \end{center}
\end{frame}

% Motivation
\begin{frame}{Motivation}
  \begin{block}{When Analytic Methods Fail}
    \begin{itemize}
      \item Complex statistics (medians, quantiles, correlations)
      \item Non-standard distributions or small samples
      \item Model misspecification or unknown sampling distributions
      \item Functions of multiple parameters
    \end{itemize}
  \end{block}

  \begin{block}{Bootstrap Idea}
    Use the data itself as an estimate of the population distribution.
    Resample with replacement to approximate the sampling distribution
    of any statistic of interest.
  \end{block}

  \begin{block}{Connection to Lesson 1}
    Bootstrap approximates the sampling distribution without knowing
    the true population parameters or distribution form.
  \end{block}
\end{frame}

% Nonparametric Bootstrap Algorithm
\begin{frame}{Nonparametric Bootstrap Algorithm}
  \begin{block}{Steps}
    \begin{enumerate}
      \item Compute statistic of interest: $\hat{\theta} = g(X_1, \dots, X_n)$
      \item For $b = 1$ to $B$:
      \begin{itemize}
        \item Draw bootstrap sample $X_1^*, \dots, X_n^* \sim$ empirical distribution
        \item Compute $\hat{\theta}^{*b} = g(X_1^{*b}, \dots, X_n^{*b})$
      \end{itemize}
      \item Use $\{\hat{\theta}^{*1}, \dots, \hat{\theta}^{*B}\}$ for inference
    \end{enumerate}
  \end{block}

  \begin{block}{Key Insight}
    The empirical distribution $\hat{F}_n$ serves as a nonparametric
    estimate of the true distribution $F$, enabling resampling without
    parametric assumptions.
  \end{block}
\end{frame}

\begin{frame}{Bootstrap Algorithm -- Visual}
  \begin{center}
    \includegraphics[width=0.95\textwidth]{figures/bootstrap_algorithm_visual.png}
  \end{center}
\end{frame}

% Bootstrap Confidence Intervals
\begin{frame}{Bootstrap Confidence Intervals}
  \begin{block}{Percentile Interval}
    \[\left[ q_{\alpha/2}^*, q_{1-\alpha/2}^* \right],\]
    where $q_p^*$ is the $p$-quantile of bootstrap replicates.
  \end{block}

  \begin{block}{Basic Interval}
    \[\left[ 2\hat{\theta} - q_{1-\alpha/2}^*, \, 2\hat{\theta} - q_{\alpha/2}^* \right].\]
    Centers the interval around $\hat{\theta}$ and uses bootstrap variance.
  \end{block}

  \begin{block}{BCa Interval (Advanced)}
    Bias-corrected and accelerated interval that adjusts for bias and
    skewness in the bootstrap distribution.
  \end{block}
\end{frame}

% Example A: Median CI
\begin{frame}{Example: Median CI (Exponential Data)}
  \begin{block}{Setup}
    $X_i \sim \Exponential(\lambda)$ i.i.d. (skewed distribution).
    Statistic: sample median $\hat{\theta} = \median(X_1, \dots, X_n)$.
  \end{block}

  \begin{block}{Challenge}
    The sampling distribution of the median is complex and depends
    on the underlying distribution, making analytic CIs difficult.
  \end{block}

  \begin{block}{Bootstrap Solution}
    Use nonparametric bootstrap to approximate the sampling distribution
    of the median and construct percentile or basic intervals.
  \end{block}
\end{frame}

\begin{frame}{Median Bootstrap --- Visual}
  \begin{center}
    \includegraphics[width=0.9\textwidth]{figures/bootstrap_median_distribution.png}
  \end{center}
\end{frame}

% Example B: Difference in Means
\begin{frame}{Example: Difference in Means}
  \begin{block}{A/B Testing Scenario}
    Compare means between two groups with potentially different variances.
    Use heights data from \texttt{shared/data/heights\_weights\_sample.csv}.
  \end{block}

  \begin{block}{Challenge}
    Welch's t-test assumes normality and provides analytic intervals,
    but bootstrap offers a robust alternative without strong assumptions.
  \end{block}

  \begin{block}{Bootstrap Approach}
    Bootstrap both groups separately and compute bootstrap distribution
    of the difference in means for inference.
  \end{block}
\end{frame}

% Number of Resamples and Practical Considerations
\begin{frame}{Practical Considerations}
  \begin{block}{Number of Bootstrap Resamples}
    \begin{itemize}
      \item $B = 1000$ often sufficient for basic intervals
      \item $B = 5000$ or more for precise quantile estimation
      \item Computational cost scales with $B \times n$
      \item Parallelization can speed up computation
    \end{itemize}
  \end{block}

  \begin{block}{Random Seeds}
    Always set random seeds for reproducibility:
    \begin{itemize}
      \item Different seeds can give slightly different results
      \item Report seeds in publications and assignments
      \item Use \texttt{rng = np.random.default\_rng(2025)}
    \end{itemize}
  \end{block}
\end{frame}

% BCa Method (Conceptual)
\begin{frame}{BCa Method (Conceptual)}
  \begin{block}{Bias Correction}
    Adjusts for bias in the bootstrap distribution when the statistic
    is not centered at the true parameter.
  \end{block}

  \begin{block}{Acceleration}
    Adjusts for skewness and heteroscedasticity in the bootstrap
    distribution using jackknife estimates.
  \end{block}

  \begin{block}{When Helpful}
    BCa intervals often provide better coverage than percentile intervals,
    especially for skewed distributions or small samples.
  \end{block}
\end{frame}

% BCa Method -- Visual Example
\begin{frame}{BCa Method -- Visual Example}
  \begin{center}
    \includegraphics[width=0.98\textwidth]{figures/bca_method_visual.png}
  \end{center}
\end{frame}

% Pitfalls and Limitations
\begin{frame}{Pitfalls and Limitations}
  \begin{block}{When Bootstrap Fails}
    \begin{itemize}
      \item Dependent data (requires block bootstrap or other methods)
      \item Very small samples ($n < 10$) may not work well
      \item Heavy-tailed distributions may need many resamples
      \item Boundary parameters (variances, correlations) need care
    \end{itemize}
  \end{block}

  \begin{block}{Best Practices}
    \begin{itemize}
      \item Always compare to parametric methods when available
      \item Check bootstrap distribution shape for anomalies
      \item Use multiple random seeds to assess stability
      \item Consider parametric bootstrap when model is trusted
    \end{itemize}
  \end{block}
\end{frame}

% Exercises
\begin{frame}{Exercises}
  \begin{enumerate}
    \item Use bootstrap to construct a 95\% confidence interval for the median of exponential data.
    \item Compare bootstrap CI for mean difference vs Welch's t-interval using heights data from \texttt{shared/data/heights\_weights\_sample.csv}.
    \item Implement studentized bootstrap for the sample mean and compare to percentile bootstrap.
    \item Investigate how bootstrap performance degrades with very small sample sizes ($n = 5, 10$).
  \end{enumerate}
\end{frame}

% Summary & References
\begin{frame}{Summary}
  \begin{block}{Key Takeaways}
    \begin{itemize}
      \item Bootstrap approximates sampling distributions via resampling
      \item Percentile and basic intervals are most common
      \item Bootstrap excels for complex statistics (medians, quantiles)
      \item Provides robust alternative to parametric methods
      \item Requires careful consideration of sample size and dependence
    \end{itemize}
  \end{block}

  \begin{center}
    \textit{Robust complement to parametric confidence intervals}
  \end{center}

  \footnotesize
  \begin{itemize}
    \item Wikipedia: Bootstrap (statistics) \url{https://en.wikipedia.org/wiki/Bootstrapping_(statistics)}
    \item Efron \& Tibshirani, \textit{An Introduction to the Bootstrap}
  \end{itemize}
\end{frame}